\section{Discussion}\label{Discussion}\thispagestyle{SectionFirstPage} % Hide headers on the first page of the section
In this paper, we try to assess how users of OSN interact differently based on their topic of discussion. To answer this question, we examine how the node-level network characteristics of degree, centrality, and clustering affect the probability of interaction and the extent to which two nodes interact in different topic-network structures. We estimate regression models on a unique data set of 2,259,717 sampled tweets on five politics-related topics: \texttt{brexit}, \texttt{populism}, \texttt{refugees}, \texttt{terrorism}, and \texttt{unemployment}.

We find some evidence for systematic differences in the direction of the effect of centrality measures between distinct topics of discussion, but not for degree and clustering metrics. In particular, we find that the receiver user's eigencentrality positively predicts the interaction between users, regardless of the topic. Nevertheless, among the pairs of interacting users, the direction of this effect varies between topics. By contrast, the sender's eigencentrality negatively influences the probability of interaction, but not the extent to which users interact between them. We hypothesize that these variations might be uniquely attributed to the different intrinsic topic characteristics. However, the results of this study cannot be taken as evidence for supporting this conjecture because of the preliminary character of this investigation.
\subsection{Managerial Implications}
Although exploratory, this study provides some insights into how users behave and interact on OSN. Since some users in the network ignite multiple conversations, each revolving around a distinct topic, it seems essential to identify key nodes in the network. Furthermore, gaining knowledge of the specific network characteristics allows for the design of effective communication strategies that can minimize the latency in the information flow.
Therefore, we propose a method for collecting Twitter data and analyzing topic-network structures that could be of interest not only to political marketers and, ultimately, to design the best strategy for efficient and well-targeted marketing campaigns on OSN.
\subsection{Limitations of This Study and Further Research}
We carry out a static analysis of tweet conversations. Studying the evolution of interaction behavior during the electoral period, perhaps in real-time, could be the object of future studies \citep[see][]{mulder:2019}.

The deficit of more and diverse network topology measures, as well as the limited number of topics taken into consideration, undermine the predictive power of this study. In particular, further research is needed to incorporate network modularity and take into account the resulting communities. Moreover, we capture topic information merely from the tracked hashtags without analyzing the content of conversation tweets. Future studies could overcome these limitations with parallel or distributed computing that allows for text analysis on such large data sets and without the need for a further sampling of data. Finally, future research could be extended to OSN other than Twitter.
