\section{Introduction}\label{Introduction}\thispagestyle{SectionFirstPage} % Hide headers on the first page of the section
The rise of synchronous computer-mediated communication (CMC) in the last decades allowed for instant sharing and access to worldwide information, which profoundly revolutionized social activity and human interaction on the Internet. Ideas, political messages, tweets, memes, and advertisements spread today on the Internet within enormous networks of nodes and edges. Using online social networks services (OSN), individuals can choose whom to engage with, constructing their own network of interactions. Albeit recommendation algorithms have taken hold in the recent years, users still actively select their information sources. This process varies according to the way the social network platform is designed – e.g., by following, subscribing, sharing, etc. – yet its rationale is almost universal.

Many recent studies on social network analysis (SNA) document these patterns of interaction. Central issues of particular interest range from the evolution of interaction behavior in social networks over time \citep[e.g.,][]{mulder:2019} to the role of message in viral marketing campaigns \citep[e.g.,][]{liu-thompkins:2012} and Twitterstorms phenomena \citep[e.g.,][]{timm:2016}.
Researchers have extensively studied single network characteristics – i.e., tie strength, centrality, and density measures, to name a few – to classify users and explain information diffusion at the individual level in social networks \citep{granovetter:1973, borgatti:2005}.
In fact, SNA has been applied to OSN in many fields, from studying the influence of word of mouth on new product launches \citep[e.g.,][]{deer:2019} to sexual relationships and couple formation \citep[e.g.,][]{ortega:2017}. While these approaches have proven successful, a deeper understanding of information diffusion at the network level can only be achieved when taking multiple measurements jointly \citep{himelboim:2017}.

This appears of utmost relevance in the light of the prominent influence that OSN exert on the public opinion nowadays \citep{kwak:2010, lerman:2010}. For instance, it might be especially worth studying how OSN are displacing more traditional media outlets during political election campaigns \citep[see, e.g.,][]{hemsley:2019, buccoliero:2020, kruikemeier:2014}.
In fact, many academics have carried out studies about the political discourse in the context of OSN, the majority of which mainly focused on two concerns: which users are the most ``influential'' (key-members identification) and how information flows throughout the network \citep[as cited in][]{bode:2016}. Previous work on information diffusion and key-members identification in social networks has been traditionally based on centrality measures \citep{freeman:1978}. However, we believe that further investigation is needed to better grasp how network characteristics, such as centrality, differ per topic in online conversations.
% no citation

In this paper, we argue that the above-mentioned network-level measurements may systematically differ among topics of conversation on OSN. We are of the view that analyzing such topic-network structures might help to better understand information dissemination. We ground our work on political conversations because of their intrinsically polarizing, and easily identifiable topics. More specifically, we focus on the 2019 European Election event because of its presence in multiple countries. Twitter can be considered suitable for these research purposes. First, as one of the foremost social network platforms that is globally used and steadily growing, it established itself as one of the most popular online political arena \citep{tumasjan:2011} and empowered politicians to share their messages broadly without the need of journalists \citep{blumler:2001}. Moreover, by providing open access to its API – with a free tier available too – it offers researchers an unprecedented opportunity to easily collect data. The retweet feature allows its users to spread information beyond the reach of their original followers, and therefore enables us to clearly reconstruct the conversation trees. Lastly, \citet{Toubia:2013} also showed that Twitter's design pushes its users to voluntarily contribute with content because of the intrinsic utility and the image-related utility, de-facto disclosing their personal information, thoughts, and experiences by tweeting, and thus providing a constant and sheer volume of data.
% because of the image utility and ... utility hypotheses?
In this study, we collect more than 20 million tweets on the 2019 European Parliament election, over the entire electoral period, from one week before (May 16, 2019) up to more than a week after the elections in all the Member States. This is a \emph{sample} of the stream of tweets in that time period, resulted from the tracking of approximately 700 different keywords\footnote{A complete list of tracked keywords can be found in the Appendix section.} via the Twitter real-time filter API\footnote{Twitter's APIs and their limitations will be thoroughly discussed in section \ref{Data}.}. We track eleven politics-related topics, chosen because of their relevance and omnipresence in most countries, with the following keywords\footnote{All the keywords have been accordingly translated into all the different languages spoken in the European Union.}: \texttt{brexit}, \texttt{eu institutions}, \texttt{fake news}, \texttt{gender equality}, \texttt{lgbt rights}, \texttt{populism}, \texttt{public debt}, \texttt{refugees}, \texttt{single market}, \texttt{terrorism}, \texttt{unemployment}.
Subsequently, we select our data to retain only contributing users to a given topic – i.e., users whose tweets initiated a conversation, as well as their retweets, replies, and @-mentions. We construct topic-network structures by obtaining nodes and edges for each topic, and calculate a set of network measurements via Gephi 0.9.2\footnote{Gephi is a visualization and exploration software for graphs and networks that is open-source and free: \url{https://gephi.org/}.}. We propose three generalized regression models to explain how information flow characteristics differ among topics, and apply them to four different topic-networks. We find evidence for systematic differences in the direction of the effect of centrality measures between distinct topics of discussion, but not for degree and clustering metrics.

Therefore, this paper will be structured as follows: in section \ref{Literature} we deepen into the extant literature. Our data collection and modeling methodology is outlined in section \ref{Data}. In section \ref{Models}, we explain our models and analysis. In section \ref{Results}, we discuss the results. Section \ref{Discussion} concludes this paper.
